\documentclass[a4paper, 11pt]{article}
\usepackage[margin=24mm]{geometry}
\usepackage{float}
\usepackage{titling}
%
\begin{document}
%
\setlength{\droptitle}{-7em}
\title{\textbf{Skydiving Formation Recognition}}
\author{Merlin Webster - 25603388 - mw16g12@soton.ac.uk \\Supervisor: Dr Jonathon S Hare - jsh2@ecs.soton.ac.uk}
\date{}
\maketitle{}
%
\thispagestyle{empty}
%
\noindent Contrary to popular belief, skydiving is a competitive and technical sport, requiring careful control of body position in order to not only remain stable, but to move around the sky in free-fall in a controlled manner.\\
A popular discipline in the sport is formation skydiving (FS). This involves multiple skydivers forming set formations in free-fall, in order to score as many points as possible. A point is awarded for each successful formation in a sequence.
\\
In order for the team to create formations in the sky, it is important that they practice the skydive multiple times on the ground. This is known as "dirt diving" and is often done using small wheeled platforms that each skydiver lays on, known as "creepers".
\\\\
A system is proposed that would record the skydivers while dirt diving, analyse the video and identify which formations they have performed.\\
For simplicity, the initial design could use a camera mounted directly above where the skydivers are dirt diving.
The next step would be to record and analyse the skydivers with a camera positioned off at $45^\circ$, a few meters above the formation.
It is common practice for a cameraman to fly near a group of skydivers as they form formations, as a video is required as evidence when judging FS competitions.
This $45^\circ$ angle would mimic the position of the cameraman when recording the skydive.
As an extension, footage from an actual skydive could be analysed.\\
However, many potential problems arise when moving to footage from a real skydive; some of which include:
\begin{itemize}
	\item The skydivers may not be at the same height relative to each other, and may overlap in the line of sight for the camera.
	\item The skydivers may fall through clouds, obscuring the view of the formation.
	\item The video background will change as the skydivers fall, making it harder to separate them from the background.
\end{itemize}

%
\noindent If the video from the cameraman was fed into the software, an FS competition could potentially be automatically judged, providing a run-down of the team's performance. This would provide a quick initial judging of the competition entry, without the need for a judge to watch the footage.
\\\\
In order for the project to be considered a success the software must be able to do the following:
\begin{itemize}
	\item The skydivers must be separated from the background.
	\item The skeleton of the skydivers must be detected in order to determine their body position.
	\item All standard static 4-way formations must be recognisable when dirt diving. The formations are outlined by the British Parachute Association (British skydiving regulatory body).
\end{itemize}

\noindent In order to complete this project C++ will be used, making use of the open source computer vision library OpenCV 3.0 in order to reduce the time required to perform standard tasks, such as loading an image file, or receiving a video feed.
Another possible project extension would be to implement a GUI in order to make the program more accessible to a FS judge. If this were implimented the open source GUI library Qt 5.3 would be used.
If more advanced file operations are required then the open source language Boost Filesystem 3.0 would be used.
All of these libraries are platform independent; this should ensure that the program is not tied to any single operating system.
\end{document}